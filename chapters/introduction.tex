% !TEX root = ../thesis.tex

\chapter{Introduction}
Questions surrounding the nature of matter and its interactions goes as far back as at least 600 BC in ancient Greece and India, but the modern study of particle physics and its experimental techniques did not arise until the turn of the 20th century.
Some argue that the study of particle physics began in 1897 when J.\ J.\ Thomson discovered the electron through his experiments on cathode rays, in which it was discovered that the rays emitted by a heated filament were actually comprised of small charged particles~\cite{GriffithsParticle}.
Later, the Rutherford scattering experiment demonstrated that the bulk of the mass and charge of an atom is concentrated in a small core we now know as the nucleus~\cite{BargerCollider}.

% More on the history of particle physics here

Since then---with theory and experiment working hand-in-hand---major advances have been made in our understanding of the fundamental building blocks of the universe.
These advances are due in large part to the developments in experimental techniques and detector technology.
Today, modern particle physics experiments are conducted in particle accelerators---massive structures in which sub-atomic particles are collided together at energies high enough to produce exotic daughter particles in collision events.
The largest of these is the Large Hadron Collider (LHC) in Geneva, Switzerland, which is 27 km in diameter and is where the Higgs boson was discovered in 2012.
Compared to early the laboratories in which particle physics was born, the LHC is a colossal machine, both in terms of expense required to build and maintain it, and the level of collaboration required to perform particle and nuclear physics experiments at the facility in the 21st century.

To that end, the LHC comes with high demands for cutting-edge hardware and software to push the frontiers of experimental particle physics, and the detectors in the facility will receive numerous upgrades over their lifetimes.
As of this writing, the LHC is currently in its second out of five `Long Shutdown' (LS) phases---prolonged periods of inactivity of the accelerator facility so that new hardware can be installed---with LS2 scheduled to conclude in March of 2021.

What follows in the first part of this work is the description, simulation, and results of a novel algorithm for reconstructing the tracks made by muons that are created in collision events in the Compact Muon Solenoid (CMS) detector at the LHC.

The Standard Model is the prevailing theory in particle physics that classifies all known elementary particles, and describes how the interact with each other via the electromagnetic, weak, and strong forces.
The theory came about during the latter half of the 20th century in order to explain observations made during early collision experiments.
