% !TEX root = ../thesis.tex

\chapter{Introduction}

Questions surrounding the nature of matter and its interactions go as far back as at least the 6th century BC in ancient Greece and India~\cite{PhysNuclphys196p}, but the modern study of particle physics and its experimental techniques did not arise until the turn of the 20th century.
Most argue that the study of particle physics began in 1897 when J.\ J.\ Thomson discovered the electron through his experiments on cathode rays, in which it was discovered that the rays emitted by a heated filament were in fact not rays at all, but were actually streams of small charged particles~\cite{GriffithsParticle}.
From this, Thomson extrapolated that atoms must be composite, and that electrons were one of the main components.
Later, the Rutherford scattering experiment demonstrated that the bulk of the mass and charge of an atom is concentrated in a small core we now know as the nucleus~\cite{BargerCollider}.
% Add sentence about completing the atom

% Add paragraph about further discoveries

Since then---with theory and experiment working hand-in-hand---major advances have been made in our understanding of the fundamental building blocks of the universe.
These advances are due in large part to the developments in experimental techniques and detector technology.
Today, modern particle physics experiments are conducted in particle accelerators---massive structures in which sub-atomic particles are collided together at energies high enough to produce exotic daughter particles in collision events.
The largest of these is the Large Hadron Collider (LHC) in Geneva, Switzerland, which is 27 km in diameter and is where the Higgs boson was discovered in 2012.
Compared to early the laboratories in which particle physics was born, the LHC is a colossal machine, both in terms of expense required to build and maintain it, and the level of collaboration required to perform particle and nuclear physics experiments at the facility in the 21st century.

This work describes the search for a new fundamental particle using data obtained by the LHC during Run 2 of data collection between 2016 and 2018.
Chapter~\ref{chap:theory} describes the theoretical background and motivation for searching for a new particle, and begins with a brief overview of the Standard Model of particle physics, later going over theoretical models beyond the Standard Model.
In chapter~\ref{chap:exp}, we go over the LHC and the Compact Muon Solenoid (CMS) detector from which the data for the search was obtained, with descriptions of the core components of the detector.
Afterwards in chapter~\ref{chap:search}, the analysis portion of the search is described, with a complete explanation of how the process was performed and what results were obtained.
Finally, in appendix~\ref{chap:TPSAppendix}, additional work that was performed towards a novel algorithm for detecting muons in the CMS detector is described.
