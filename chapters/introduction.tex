% !TEX root = ../thesis.tex

\chapter{Introduction}
\label{chap:intro}

% Origins of the study of particle physics
Questions surrounding the nature of matter and its interactions go as far back as at least the 6th century BC in ancient Greece and India~\cite{PhysNuclphys196p}, but the modern study of particle physics and its experimental techniques did not arise until the turn of the 20th century.
Most argue that the study of particle physics began in 1897 when J.\ J.\ Thomson discovered the electron through his experiments on cathode rays, in which it was discovered that the rays emitted by a heated filament were in fact not rays at all, but were actually streams of small charged particles~\cite{GriffithsParticle}.
From this, Thomson extrapolated that atoms must be composite, and that electrons were one of the main components.
Later, the Rutherford scattering experiment demonstrated that the bulk of the mass and charge of an atom is concentrated in a small core we now know as the nucleus~\cite{BargerCollider}, which led to the discovery of the proton.
Then in 1932, James Chadwick discovered the neutron and completed the atomic picture of matter~\cite{weinberg2003discovery}.

% Discussion on the significance of the photon
Meanwhile, in 1900 Max Planck proposed a solution to the ultraviolet catastrophe, which was a prediction from classical physics that the total power emitted by an ideal black body at thermal equilibrium should be infinite~\cite{schroeder2013introduction}.
Planck resolved the problem by assuming that electromagnetic radiation emitted by a black body can only come in discrete energy packets, which resulted in a power spectrum that matched with experiments at the time.
While Planck had no explanation for why the energy came in discrete packets, Einstein took this notion a step further in 1905 by instead insisting that the electromagnetic field itself was quantized, and that the discrete packets were in fact particles~\cite{doi:10.1002/andp.19053220607}, which we now know as photons (denoted by the symbol $\gamma$).
In addition to explaining the observed behavior of the black body power spectrum, Einstein's theory also provided an explanation for the photoelectric effect, for which he eventually received the 1921 Nobel Prize in Physics~\cite{NobelPrize:1921-Physics}.

% The first mesons
Despite the success of the Rutherford scattering experiment and the subsequent discovery of the proton, the question of how protons---which are positively charged and should repel each other via the electromagnetic force---could be bound together was a lingering issue.
Later in the 1930's and 40's, experiments with cosmic rays led to the discovery of the $\mu$ lepton and the $\pi$ meson~\cite{Lattes_1947}.
Initial experiments were unable to distinguish between the two particles, and it was thought that the $\pi$ meson was the mediator of a new force that binds the protons in the nucleus of an atom together.
This force came to be known as the strong nuclear force, and although the neither the $\mu$ lepton nor the $\pi$ meson turned out to be the mediators of this force, later experiments involving deep scattering to probe atomic nuclei would give indirect evidence for the existence of the gluon ($g$)---the true mediator of the strong force.

% Discovery of antiparticles and neutrinos
Just prior to the discovery of the $\pi$ meson, the positron was discovered in 1931~\cite{doi:10.1119/1.1937627}, which was the first example of an antiparticle---an oppositely charged twin to a particle.
As it turned out, one key prediction of quantum field theory is that every particle is associated with a corresponding antiparticle, which has the same mass but opposite physical charge.
Another landmark discovery around the same time was the discovery of neutrinos, which were postulated in 1930 by Wolfgang Pauli to explain beta decays in atomic nuclei~\cite{Reines-Cowan}.
Their existence was deduced in order to reconcile the observed energy spectrum of the electron emitted during the decay process $n\to p^++e^-+\bar{\nu}$, as the results were kinematically inconsistent with a two-body decay.
It was also later experimentally verified in 1962 that there was more than one neutrino, and each is associated to a corresponding lepton~\cite{doi:10.1119/1.1976245}.

% Development of the quark model
By the 1950's more particles were discovered through cosmic ray experiments, and the first particle accelerators started to come online.
At this point, the field of particle physics was marred with the problem of finding structure to all the new particles that were being discovered.
Many new mesons were discovered during this period and into the 1960's, such as kaons, $\eta$'s, and $\Lambda$'s~\cite{CosmicRay,PhysRevLett.7.421}.
Eventually this issue of categorizing the new particles was resolved in 1964 with the development of the quark model by Murray Gell-Mann and George Zweig, who both independently proposed that hadrons (baryons and mesons) are all composite and made up of elementary particles called quarks~\cite{Gellmann1964214,Zweig:1981pd}.
The original quark model only had three quarks: the up quark ($u$), the down quark ($d$), and the strange quark ($s$).
The model also proposed that every baryon is made of three quarks, and every meson is made of a quark and an antiquark.
This neatly explained the properties of all as of yet observed baryons and mesons, and it superseded the original classification scheme known as the Eightfold Way.
However, this model was initially met with skepticism, but deep inelastic scattering experiments in the 1960's and 1970's would confirm that the proton indeed has substructure consistent with the quark model~\cite{PhysRevLett.23.930,cernQuarks}.

% The November revolution
Then in 1974, the discovery of the $J$/$\psi$ meson by Samuel Ting at Brookhaven National Lab and Burton Richter at the Stanford Linear Accelerator Facility would give further evidence for the quark model, as it turned out to be a bound state of a new quark---called the charm quark ($c$)---and its antiquark~\cite{PhysRevLett.33.1406,PhysRevLett.33.1404}.
Later experiments in the following years would also produce baryons with this new quark, along with a new lepton called the $\tau$ lepton, a neutrino associated with the $\tau$ called $\nu_\tau$~\cite{Perl:1976rz}, and a new quark called the bottom quark ($b$) after discovering the upsilon meson~\cite{1977PhRvL..39..252H}.
By this point, there were six leptons ($e$, $\mu$, $\tau$, $\nu_e$, $\nu_\mu$, $\nu_\tau$), and five quarks ($u$, $d$, $c$, $s$, $b$).
It seemed reasonable to think that there would perhaps be a sixth quark, bringing the number of quarks in line with the number of leptons.
While this turned out to be true, the discovery of the sixth quark, known as the top quark ($t$), would not come until much later in 1995 at the Tevatron~\cite{PhysRevLett.74.2626}.

% Development of the Standard Model
Another remaining issue was the lack of experimental evidence for a mediator of the weak nuclear force, which is responsible for nuclear decay.
In 1967, Steven Weinberg's work on electroweak unification predicted the existence and the masses of the mediators of the weak force, known as the $W^\pm$ and $Z$ bosons~\cite{PhysRevLett.19.1264}.
His work, along with Sheldon Glashow and Abdus Salam, would lead to the three being awarded the 1979 Nobel Prize in Physics~\cite{NobelPrize:1979-Physics}.
The discovery of the $W^\pm$ and $Z$ bosons would then occur at the European Organization for Nuclear Research (CERN) in 1983, with the observed masses falling well within the predicted ranges given by electroweak theory~\cite{Arnison1983103,Arnison1983398}.
These crucial developments helped lead to widespread acceptance of the Standard Model (SM) of particle physics, and with the discovery of the top quark in 1995, the sub-atomic picture of matter was in a much tidier state than compared to the beginning of the 20th century: there are six quarks, six leptons, and four gauge bosons ($g$, $\gamma$, $W^\pm$, $Z$) that mediate the strong, weak, and electromagnetic interactions.

% Discovery of the Higgs
One particle predicted by the Standard Model still eluded experiments up to this point.
A key prediction of the Standard Model is that of the Higgs mechanism, through which the gauge bosons for the weak interaction have their masses imparted onto them via the mechanism of spontaneous symmetry breaking.
This required the existence of a new boson known as the Higgs, and despite the fact that the Higgs mechanism was developed during the formulation of the Standard Model in the 1960's~\cite{PhysRevLett.13.508}, it would not be until much later that the particle itself was discovered.
After the discovery of the top quark, the Higgs would elude physicists for almost two decades before being discovered at the Large Hadron Collider (LHC) at CERN in 2012~\cite{20121,201230}.

% Shortcomings of the standard model and motivation for thesis work
The discovery of the Higgs boson marked another triumph for the predictive power of the Standard Model.
But despite all of its success over the last few decades, there are still longstanding issues that the Standard Model is currently incapable of addressing.
While it describes the strong, weak, and electromagnetic interactions very well at the energy scales present experiments have access to, one major shortcoming of the Standard Model is that it does not include gravitational interactions.
In fact, the Standard Model is incompatible with general relativity~\cite{Macias200899}.
Efforts to formulate a quantum theory of gravity are still ongoing, and the effects of gravitational interactions at such small scales would be extremely difficult to observe in an experimental setting.
Another issue is that the baryonic matter that the Standard Model describes only makes up about 5\% of the observed universe, while another 27\% is dark matter, and the remaining 68\% is dark energy~\cite{Planck2013}---neither of which the Standard Model accounts for.
Furthermore, the Standard Model predicts that neutrinos should be massless, but experiments observing neutrino oscillations show that they do have mass~\cite{Ahmad_2001}.
Finally, the Standard Model does not account for the observed matter-antimatter asymmetry, as it predicts that matter and antimatter should have been created in equal amounts at the beginning of the universe~\cite{astroParticle}.

Efforts to search for new physics beyond the Standard Model (BSM) are ongoing, and many theories such as supersymmetry or grand unified theories predict exotic new particles that may be produced in particle accelerators.
In addition to verifying the existence of the Higgs boson, an equally important goal of the LHC at CERN is to search for evidence of such exotic particles, or to find any observations that deviate from the predictions of the Standard Model~\cite{doi:10.1080/0010751031000077378,Kanti}.

% Thesis overview
This work describes the search for a new fundamental particle using data obtained by the LHC during Run 2 of data collection between 2016 and 2018.
Chapter~\ref{chap:theory} describes the theoretical background and motivation for searching for a new particle, and begins with a brief overview of the Standard Model of particle physics, later going over theoretical models beyond the Standard Model relevant for the search.
In chapter~\ref{chap:exp}, we go over the LHC and the Compact Muon Solenoid (CMS) detector from which the data for the search was obtained, with descriptions of the core components of the detector.
Afterwards in chapter~\ref{chap:search}, the analysis portion of the search is described, with a complete explanation of how the process was performed and what results were obtained.
Finally, in appendix~\ref{chap:TPSAppendix}, additional work that was performed towards a novel algorithm for detecting muons in the CMS detector is described.
