% !TEX root = ../thesis.tex

\chapter{Introduction}

% Origins of the study of particle physics
Questions surrounding the nature of matter and its interactions go as far back as at least the 6th century BC in ancient Greece and India~\cite{PhysNuclphys196p}, but the modern study of particle physics and its experimental techniques did not arise until the turn of the 20th century.
Most argue that the study of particle physics began in 1897 when J.\ J.\ Thomson discovered the electron through his experiments on cathode rays, in which it was discovered that the rays emitted by a heated filament were in fact not rays at all, but were actually streams of small charged particles~\cite{GriffithsParticle}. % Possible check for a different source
From this, Thomson extrapolated that atoms must be composite, and that electrons were one of the main components.
Later, the Rutherford scattering experiment demonstrated that the bulk of the mass and charge of an atom is concentrated in a small core we now know as the nucleus~\cite{BargerCollider}, which led to the discovery of the proton.
Then in 1932, James Chadwick discovered the neutron and completed the atomic picture of matter. % Citation

% Discussion on the significance of the photon
Meanwhile, in 1900 Max Planck proposed a solution to the ultraviolet catastrophe, which was a prediction from classical physics that the total power emitted by an ideal black body at thermal equilibrium should be infinite. % Citation
Planck resolved the problem by assuming that electromagnetic radiation emitted by a black body can only come in discrete energy packets, which resulted in a power spectrum that matched with experiments at the time.
While Planck had no explanation for why the energy came in discrete packets, Einstein took this notion a step further in 1905 by instead insisting that the electromagnetic field itself was quantized, and that the discrete packets were in fact particles, which we now know as photons. % Citation
In addition to explaining the observed behavior of the black body power spectrum, Einstein's theory also provided an explanation for the photoelectric effect, for which he eventually received the 1921 Nobel Prize. % Citation

% The first mesons
Despite the success of the Rutherford scattering experiment and the subsequent discovery of the proton, the question of how protons---which are positively charged and should repel each other via the electromagnetic force---could be bound together was a lingering issue.
Later in the 1930's and 40's, experiments with cosmic rays led to the discovery of the $\mu$ lepton and the $\pi$ meson. % Citation
Initial experiments were unable to distinguish between the two particles, and it was thought that the $\pi$ meson was the mediator of a new force that binds the protons in the nucleus of an atom together.
This force came to be known as the strong nuclear force, and although the neither the $\mu$ lepton nor the $\pi$ meson turned out to be the mediators of this force, later experiments involving deep scattering to probe atomic nuclei would give indirect evidence for the existence of the gluon---the true mediator of the strong force.

% Discovery of antiparticles and neutrinos
Just prior to the discovery of the $\pi$ meson, the positron was discovered in 1931, which was the first example of an antiparticle---an oppositely charged twin to a particle. % Citation
As it turned out, one key prediction of quantum field theory is that every particle is associated with a corresponding antiparticle, which has the same mass but opposite physical charge.
Another landmark discovery around the same time was the discovery of neutrinos, which were first observed in 1930 during beta decays in atomic nuclei. % Citation
Their existence was deduced in order to explain the observed energy spectrum of the electron emitted during the decay process $n\to p^++e^-+\bar{\nu}$, as the results were kinematically inconsistent with a two-body decay.
It was also later experimentally verified in 1962 that there was more than one neutrino, and each is associated to a corresponding lepton. % Citation

% Development of the quark model
By the 1950's more particles were discovered through cosmic ray experiments, and the first particle accelerators started to come online.
At this point, the field of particle physics was marred with the problem of finding structure to all the new particles that were being discovered.
Many new mesons were discovered during this period and into the 1960's, such as kaons, $\eta$'s, and $\Lambda$'s. % Citation
Eventually this issue of categorizing the new particles was resolved in 1964 with the development of the quark model by Murray Gell-Mann and George Zweig, who proposed that hadrons (i.e., barons and mesons) are all composite and made up of elementary particles called quarks. % Citation
However, this model was initially met with skepticism, but deep inelastic scattering experiments in the 1960's and 1970's would confirm that the proton indeed has substructure consistent with the quark model. % Citation

% The November revolution

% Development of the standard model

% Discovery of the Higgs

% Shortcomings of the standard model and motivation for thesis work

% Thesis overview
This work describes the search for a new fundamental particle using data obtained by the LHC during Run 2 of data collection between 2016 and 2018.
Chapter~\ref{chap:theory} describes the theoretical background and motivation for searching for a new particle, and begins with a brief overview of the Standard Model of particle physics, later going over theoretical models beyond the Standard Model.
In chapter~\ref{chap:exp}, we go over the LHC and the Compact Muon Solenoid (CMS) detector from which the data for the search was obtained, with descriptions of the core components of the detector.
Afterwards in chapter~\ref{chap:search}, the analysis portion of the search is described, with a complete explanation of how the process was performed and what results were obtained.
Finally, in appendix~\ref{chap:TPSAppendix}, additional work that was performed towards a novel algorithm for detecting muons in the CMS detector is described.
